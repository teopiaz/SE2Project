\documentclass[english]{article}
\usepackage[T1]{fontenc}
\usepackage[utf8]{luainputenc}
\setcounter{secnumdepth}{4}
\setcounter{tocdepth}{4}
\usepackage{color}
\usepackage{babel}
\usepackage{array}
\usepackage{graphicx}
\usepackage{setspace}
\usepackage{todonotes}
\usepackage[unicode=true,
 bookmarks=true,bookmarksnumbered=true,bookmarksopen=false,
 breaklinks=false,pdfborder={0 0 1},backref=false,colorlinks=false]
 {hyperref}
\hypersetup{pdftitle={RASD},
 pdfauthor={Piazzolla Matteo Michele - Millimaggi Andrea},
 pdfsubject={RASD Documentation}}

\makeatletter

\providecommand{\tabularnewline}{\\}

%%%%%%%%%%%%%%%%%%%%%%%%%%%%%% Textclass specific LaTeX commands.
\newcommand{\lyxrightaddress}[1]{
\par {\raggedleft \begin{tabular}{l}\ignorespaces
#1
\end{tabular}
\vspace{1.4em}
\par}
}


\newcounter{requirement}
\setcounter {requirement} {0} 

\newcommand{\reqcounter}{
\stepcounter{requirement}
\item \textbf{[R-\arabic{requirement}]}
}




\makeatother

\begin{document}

\title{\includegraphics[scale=0.4]{img/polimi}\\
Computer Science and Engineering}

\begin{doublespace}

\author{A.A. 2016/2017\\
Software Engineering 2 Project: \\
\\
{\LARGE{}``PowerEnJoy''}\textbf{}\\
\\
\textbf{R}equirements \textbf{A}nalysis and \textbf{S}pecification
\textbf{D}ocument\\
}
\end{doublespace}

\maketitle

\lyxrightaddress{Prof.Luca Mottola\\
\\
Matteo Michele Piazzolla Matr. 878554\\
Andrea Millimaggi Matr. 876062}

\newpage{}
\listoftodos
\newpage{}

\tableofcontents{}

\newpage{}
\listoffigures

\newpage{}

\section{Introduction}


\subsection{Purpose}
\todo{forse sbagliato, nelle slide c'è scritto di mettere qui i goals}

This document is the RASD (Requirement Analysis and Specification Document). The goal of this document is to describe the functional and non-functional requirements of the system in order to show the constraints on the software that the developer has to take into account. This document can have a legal value and can be used as a contractual basis between the customers and the developers.

\subsection{Scope}
The system the has to be developed is called \emph{PowerEnJoy}. It is a digital system for the management of a car-sharing service that only employs electric cars. In particular the aim is to develop a mobile application that allows the user to pick up and use electric cars in the areas reached by the service. 

\subsection{Definitions, Acronyms, Abbreviations}


\subsubsection{Definitions}
\begin{itemize}
	\item User: is a general person who uses ``PowerEnJoy''.
	\item Registered User: a user that have completed the registration process.
	\item Guest: a visitor, not authenticated user.
	\item Driver: authenticated user that drive the requested car.
	\item Car reservation: allocation of a car for a definited amount of time.

	\todo {[Andrea] Ho pensato di divedere così le aree. Nella safe area sono inclusi i parcheggi normali}

	\item Safe area: area in which the User can find cars, pick them up and park them.
	\item Special parking area: area in which the user can park the car and also charge it plugging it into the power grid. 
	\item Car-Sharing Area: area in which users can find and park a car of the car-sharing service.
	
\end{itemize}

\subsubsection{Acronyms}
\todo[inline]{Copiati da controllare mentre stiliamo il documento}
\begin{itemize} 
	\item RASD: Requirements Analysis and Specification Document.
	\item DB: Database.
	\item DBMS: Database management system.
	\item API: Application Programming Interface.
	\item OS: Operating System.
	\item JVM: Java Virtual Machine.
	\item J2EE: Java Enterprise Edition.
\end{itemize}

\subsubsection{Abbreviations}
\begin{itemize}
	\item {[}Gn{]}: n-goal.
\end{itemize}


\subsection{Reference documents}
\begin{itemize}
	\item IEEE Std 830-1998 IEEE Recommended Practice for Software Requirements Specifications.
	\item IEEE Std 1016 tm -2009 Standard for Information Technology-System Design-Software Design Descriptions.
	\item ISO/IEC/IEEE International Standard - Systems and software engineering -- Life cycle processes --Requirements engineering
\end{itemize}

\subsection{Document Overview}
The structure of this document follows the IEEE standard for the editing of a RASD document. In particular the structure is:

\begin{itemize}
	\item Introduction: it provides a general description of the document and the system that has to be developed
	\item Overall Description: it provides a general description of all the things that  affects the work on the project and the elaboration of the requirements. So it gives information about hardware and software choices and availability, the main functions of the product, the description of the people at whom the product is addressed, the constraints that will limit the work of the developer, the assumptions done before the development and the dependencies that the use of the product will have from hardware and software entities
	\item Specific Requirements: it provides a detailed description of the requirements of the product
\end{itemize}


\subsection{Given Problem}

\subsection{Proposed System}


\subsection{Stakeholders}
\begin{itemize}
	\item Citizens
	\todo {cancellati Electric cars and electricity providers (ha detto a lezione che non sono stakeholders)}
	
	
\end{itemize}



\subsection{Actors}
\todo[inline]{aggiungiamo la macchina e le safe area?}
The entities identified in the problem description are the following: 
\begin{itemize}
	\item GUEST: is a visitor, someone that isn't signed-up yet. He can visualize
	"PowerEnJoy'' Web site and download the mobile app but he cannot
	access to any service.
	\item USER: is a client signed-up, via Web or mobile app. He accesses
	"Passenger Area'' and, after being successfully logged in, he can see
	his personal section, modify his personal information. He can also
	access to all services,view a map of available cars, make car reservations, modify or
	cancel them.
\end{itemize}

\subsection{Goals}
\todo{In teoria la lista dei goal deve essere presente}
\begin{itemize}
	\item {[}G1{]} Allow guests to register to "PowerEnJoy'' service through
	both website or mobile application.
	\item {[}G2{]} Allow registered users to login and logout.
	\item {[}G3{]} Allow authenticated users to modify their personal details
	and payment informations, such as payment method and payment records.
	\item {[}G4{]} Allow authenticated users to view a list of available cars
	in a configurable radius of distance from user position or specified
	address.
	
	\todo{[Andrea] Aggiunti Goal da 5 a 9}	

	\item {[}G5{]} Allow authenticated users to view available cars on a map in the application.
	\item {[}G6{]} Allow authenticated users to view special parking areas on a map in the application.
	\item {[}G7{]} Allow authenticated users to view special parking areas while driving.	
	\item {[}G8{]} Allow authenticated users to view the safe areas on a map in the application.
	\item {[}G9{]} Allow authenticated users to view the safe areas while driving.
		
	\item {[}G10{]} Allow authenticated users to request for pick up an available
	car for a limited time and unlock it when the User is in proximity.
	\item {[}G11{]} Allow authenticated users to drive the unlocked car inside
	the border of the city.
	\item {[}G12{]} Allow driver to be constantly informed about the service through a screen on the car.
	\item {[}G13{]} Allow driver to get discount in base of pick up/parking
	position of the ar, number of passengers and status of the battery.
\end{itemize}

\pagebreak{}



\section{Overall Description}



\subsection{Product perspective}\todo[inline]{(further details on the shared phenomena and a domain model{-->}class diagrams and statecharts)}
The system that we are going to develop is made of several parts. 
\begin{itemize}
	\item A Mobile Application usable from every Tablet or Smartphone that has access to an internet connection. The application will be available on Android and iOS. 
	\item A Website where a user can find information about the service.
	\item A Server backend to manage the service.
\end{itemize}

\subsubsection{System interfaces}
The system will depend on a device installed onboard every 'PowerEnJoy car. That onboad device will communicate with the user and via internet to the server.

\subsubsection{User interfaces}
The available interfaces will be:
\begin{itemize}
	\item the mobile application
	\item the website
\end{itemize}
The website will be implemented with a responsive design to adapt to all most common screen aspect and resolution with clear and minimal UI to favorite accessibility.
The mobilie application will be graphically similar to the website with a few views.
The use of the website or the mobile application won't be necessary during driving. 

\subsubsection{Hardware interfaces}
The main hardware interaction concerns the geolocalization feature, and in particular the GPS hardware on the smartphone device.
The application need an internet connection so a smart device with a 3G or LTE connectivity is required.

\subsubsection{Software interfaces}
To provide the best use of the service through the website an HTML5 compliant browser is required and an up to-to-date version of Android or iOS operating system for the mobile application.





\subsection{User characteristics}
Everyone who has a car license will be able to register to the car-sharing service and drive one of the electric car available.
Because of the variety of people that will use the service, the mobile application must be simple and user-friendly, so that anyone can use it even if it has a little knowledge about the use of a mobile device.

\subsection{Constraints}



\subsection{Assumptions}
\todo{Da completare!}
\begin{enumerate}
	\item Cars can be parked in every area of the city where the parking is allowed, either it is free or chargeable. The company 		                    takes care of the payment for chargeable parkings. 
	\item Every car can be uniquely recognized by his plate number.
	\item The position of the users are well-known thanks to GPS.
	\item The user is the one and only who drives the cars he reserves.
	\item Any user who reserves a car has enough money on his credit card to pay the travel.
	\item Every car is periodically checked to ensure proper operation.
\end{enumerate}

\subsection{Regulatory policies}
It’s user responsibility to ensure that the use of the system complies with the local laws and policies. If the user register to the service must allow for the permission to acquire, store and process personal data and web cookies. The system must offer to the user the possibility to delete the account and all the personal data.


\subsubsection{Possible Future Implementations}
Depending on the success that the developed system will have, it will be possible to extend the car-sharing to Scooters and Mini-Cars, in order to reach also younger customers.
\\ \noindent 
Another possibility for the future is to allow users to get in touch with strangers and pick them up in the middle of the travel.
\todo {da sistemare la seconda frase} 
\pagebreak{}

\subsection{Product requirements}
\subsubsection{Functional Requirements}

\begin{itemize}
	\item User registration
	\item Login
	\item Check for cars availability using user's position
	\item Check for cars availability using address
	\item Reserve a car
	\item Cancel a reservation
	\item Check the amount paid for a travel
	\item Check the log of reservations done
	\item View the details of cars available (level of the charge)
	
	
\end{itemize}



\pagebreak

\section{Specific Requirements}


\subsection{Functional Requirement}

With these requirements are defined features and functions with which
the user will interact directly.



\subsubsection{{[}G1{]} Allow guests to register to ``PowerEnJoy'' service through both website or mobile application.}
\begin{itemize}
\reqcounter The system shall allow registration only if the patent license number is provided.
\reqcounter The system shall provide a home page in which a guest user must be
able to know what the service is. 
\reqcounter The system shall demand the User to read and accept all term of use of the service.
\end{itemize}

\subsubsection{{[}G2{]} Allow registered users to login and logout.}
\begin{itemize}
\reqcounter The system shall show a sign-up page with a login form and a logout button if the User is already logged in.
\reqcounter The system shall validate any input in both client and server side.
\reqcounter The system shall prevent anyone from logging more than once at a time.
\reqcounter The system shall show and error message in case of wrong credential.
\reqcounter The system shall provide a password recovery procedure.
\reqcounter The system shall prevent bruteforce attack limiting the number of try per IP address.
\reqcounter The system shall redirect the User to the homepage as a guest after the logout.
\end{itemize}



\subsubsection{{[}G3{]} Allow authenticated users to modify their personal details and payment informations.}
\begin{itemize}
	\reqcounter The system shall show a an account page with all the information that the User sent during 
	account registration.
	\reqcounter The system shall ask for the password before saving the edited details.
	\reqcounter The system shall send and email to notify the changes.
\end{itemize}



\subsubsection{{[}G4{]} Allow authenticated users to view a list of available car in a configurable radius of 
	distance from user position or specified address.}
\begin{itemize}
	\reqcounter The system shall show on a web page or a view in the application the position of the available car.
	\reqcounter The system shall allow the User to set a search radius.
	\reqcounter The system shall ask the User for the GPS position or to provide an address.
\end{itemize}


\subsubsection{{[}G5{]} Allow authenticated users to request for pick up an available car for a limited time and
	 unlock it when the User is in proximity.}
\begin{itemize}
	\reqcounter The system shall not allow the User to request multiple cars at the same time.
	\reqcounter The system shall check if the payment information are valid and show an error if not.
	\reqcounter The system shall automatically cancel a car reservation if the User don't unlock the car in one hour without charging the User.
	\reqcounter The system shall check for the User position and if the distance from the car is less than 50 meters
						allow to unlock the car.
\end{itemize}

\subsubsection{{[}G6{]} Allow authenticated users to drive the unlocked car inside the border of the city.}
\begin{itemize}
	\reqcounter The system shall inform the User if she is driving outside the city area.
	\reqcounter In case of driving outside the city area the system shall charge a 100\%  more of the current charge.
\end{itemize}



\subsubsection{{[}G7{]} Allow driver to be constantly informed about the service through a screen on the car.}

\begin{itemize}
	\reqcounter The system shall inform the User if she is driving outside the city area.
	\reqcounter The system shall inform the User of the amount that will be charged.
	\reqcounter The system shall inform the User of the position of charging areas nearby the current car position.
\end{itemize}


\subsubsection{{[}G8{]} Allow driver to get discount in base of pick up/parking
	position of the ar, number of passengers and status of the battery.}\todo[inline]{Sistemare, copiati dal file del progetto}
\begin{itemize}
	\reqcounter The system shall detects the User took at least two other passengers onto the car and applies a discount of 10\% on the last ride.
	\reqcounter The system shall check if a car is left with no more than 50\% of the battery empty, the system applies a discount of 20\% on the last ride.
	\reqcounter If a car is left at special parking areas where they can be recharged and the user takes care of plugging the car into the power grid, the system shall applies a discount of 30\% on the last ride.
\end{itemize}


\subsection{Non-Functional Requirement}

\begin{itemize}
	\item PowerEnJoy has to be available 24h/7d.
	\item Both mobile application and website must be reactive and usable.
	
\subsubsection{Reliability}%affidabilità

The service has to guarantee an 24h/7d availability. Components of the project code will be tested after the implementation phase to ensure that they are functional. 
All the critical software bugs found must be patched in at least 48h.
If a Backend API change must guarantee support for client that implement older API.
All the system data must be constantly backed up to assure the data recovery in case of fault.


\subsubsection{Performance}

\subsubsection{Usability}
\begin{itemize}
	\item The User must be able to use the system with mouse, keyboard and touchscreen.
	\item The User must be able to choose from several language.
	\item At the first login the system must provide a simple and skippable 3-views tutorial of the service.
	
\end{itemize}

\begin{figure}[t]
	\centering
\includegraphics[scale=0.4]{img/webhome}
	\caption{PowerEnJoy website homepage}
\end{figure}




\subsubsection{Security}

The system must ensure that all data is protected from unauthorized
access. Password should be saved on a DB hashed and salted.
Every input from the user and every request must be sanitized 
\iffalse
It must support communication protocols used by web applications and
devices on or within the network edge. These include network transport
protocols such as HTTP and SSL/TLS, plus proxy support.
\fi
\end{itemize}

\pagebreak{}




\subsubsection{Performance}


\subsubsection{Maintainability and Portability}

\pagebreak{}




\subsubsection{Network connections}


\subsubsection{Cookies}



\subsubsection{Hardware requirement}
\todo[inline]{Matteo: da approfondire/inventare}
The communication between client and server is done using HTTPS REST API encrypted with TLS connections.
Nginx webserver and a large use of caching.
Google Maps API.



\subsubsection{Concurrent operations}

\section{ORE}\todo{da mettere poi in un file a parte}
23/10/16 
Matteo 3h 
Andrea XX\\
24/10/16 
Matteo 3h\\
25/10/16 
Matteo 1h\\
01/11/16
Matteo 1h\\
\end{document}
