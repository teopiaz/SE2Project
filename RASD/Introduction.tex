
\section{Introduction}


\subsection{Purpose}
\todo{forse sbagliato, nelle slide c'è scritto di mettere qui i goals}

This document is the RASD (Requirement Analysis and Specification Document). The goal of this document is to describe the functional and non-functional requirements of the system in order to show the constraints on the software that the developer has to take into account. This document can have a legal value and can be used as a contractual basis between the customers and the developers.

\subsection{Scope}
The system the has to be developed is called \emph{PowerEnJoy}. It is a digital system for the management of a car-sharing service that only employs electric cars. In particular the aim is to develop a mobile application that allows the user to pick up and use electric cars in the areas reached by the service. 

\subsection{Definitions, Acronyms, Abbreviations}


\subsubsection{Definitions}
\begin{itemize}
	\item User: is a general person who uses ``PowerEnJoy''.
	\item Registered User: a user that have completed the registration process.
	\item Guest: a visitor, not authenticated user.
	\item Driver: authenticated user that drive the requested car.
	\item Car reservation: allocation of a car for a defined amount of time.
	\item Safe area: area in which the User can find cars, pick them up and park them.
	\item Special parking area: area in which the user can park the car and also charge it plugging it into the power grid. 

	
\end{itemize}

\subsubsection{Acronyms}
\todo[inline]{Copiati da controllare mentre stiliamo il documento}
\begin{itemize} 
	\item RASD: Requirements Analysis and Specification Document.
	\item DB: Database.
	\item DBMS: Database management system.
	\item API: Application Programming Interface.
	\item OS: Operating System.
	\item JVM: Java Virtual Machine.
	\item J2EE: Java Enterprise Edition.
\end{itemize}

\subsubsection{Abbreviations}
\begin{itemize}
	\item {[}Gn{]}: n-goal.
\end{itemize}


\subsection{Reference documents}
\begin{itemize}
	\item IEEE Std 830-1998 IEEE Recommended Practice for Software Requirements Specifications.
	\item IEEE Std 1016 tm -2009 Standard for Information Technology-System Design-Software Design Descriptions.
	\item ISO/IEC/IEEE International Standard - Systems and software engineering -- Life cycle processes --Requirements engineering
\end{itemize}

\subsection{Document Overview}
The structure of this document follows the IEEE standard for the editing of a RASD document. In particular the structure is:

\begin{itemize}
	\item Introduction: it provides a general description of the document and the system that has to be developed
	\item Overall Description: it provides a general description of all the things that  affects the work on the project and the elaboration of the requirements. So it gives information about hardware and software choices and availability, the main functions of the product, the description of the people at whom the product is addressed, the constraints that will limit the work of the developer, the assumptions done before the development and the dependencies that the use of the product will have from hardware and software entities
	\item Specific Requirements: it provides a detailed description of the requirements of the product
\end{itemize}


\subsection{Given Problem}

\subsection{Proposed System}


\subsection{Stakeholders}
\begin{itemize}
	\item Citizens
	
\end{itemize}



\subsection{Actors}
\todo[inline]{aggiungiamo la macchina e le safe area?}
The entities identified in the problem description are the following: 
\begin{itemize}
	\item GUEST: is a visitor, someone that isn't signed-up yet. He can visualize
	"PowerEnJoy'' Web site and download the mobile app but he cannot
	access to any service.
	\item USER: is a client signed-up, via Web or mobile app. He accesses
	"Passenger Area'' and, after being successfully logged in, he can see
	his personal section, modify his personal information. He can also
	access to all services,view a map of available cars, make car reservations, modify or
	cancel them.
\end{itemize}

\subsection{Goals}
\todo{In teoria la lista dei goal deve essere presente}
\begin{itemize}
	\item \goal{1}
	\item \goal{2}
	\item \goal{3}
	\item \goal{4}
	
	\todo{[Andrea] Aggiunti Goal da 5 a 7}	
	
	\item \goal{5}
	\item \goal{6}
	\item \goal{7}
	\todo{Forse meglio dividere G10 in 2 (una per la request e uno per lo sblocco}
	
	\item \goal{8}
	\item \goal{9}
	\item \goal{10}
	\item \goal{11}
	
	\todo{[Andrea] Aggiunti G11 e G10, specificano meglio le informazioni riguardanti il servizio del G12-->forse di può togliere G12 e tenerne 2 più specifici}
	\item \goal{12}
	\todo{[Andrea] Riformulato il G15, tolta la posizione in cui si prende la macchina come discriminante per lo sconto perchè
		non è scritto nell'assignment}
	\item \goal{13}
	\item \goal{14}
	\item \goal{15}
	\item \goal{16}
	\item \goal{17}
\todo{[Andrea] Aggiunti Goals da 13 a 17, non sicurissimo che siano giusti}
\end{itemize}



\pagebreak{}


