\section{ARCHITECTURAL DESIGN}
\subsection{Overview} 
In this section is provided a complete overview of all the system components, from the logical level to the physical level of the PowerEnJoy service.


\subsection{Component view} 
From an high level prospective the service is composed by different modules allowing development and maintenance flexibility.
This modularity allow also scalability for future service expansion, \todo{espandere il concetto}

\\The main components of the system are:
\begin{itemize}
\item{\textbf{Database:}} the data layer of the service. All the persistent data will be stored in this layer.
\item{\textbf{Application Server:}} this component will manage the applicative logic.
\item{\textbf{Web Server:}} this layer is the interface with the world. It will manage the requests from the users and the website.
\item{\textbf{Mobile Application:}} this component will be a simple view that generate request and show data. \todo{espandere xD, oggi zero fantasia}
\item{\textbf{Car OnBoard Device:}} this is the module mounted on the car. It will communicate with the server and with the car ECU.
\end{itemize}

	\begin{figure}[H]	
	\centering
	\includegraphics[scale=0.5]{img/architecture_diagram}
	\caption{Architecture Diagram}
\end{figure}



\todo{Queste cose vanno nell'overview (descrizione di alto livello)}

\todo{Bisogna scegliere se usare un'architettura a 3 o a 4 tier. 3--> web server e application logic insieme, 4--->separate}


\subsection{Deployment view}
\subsection{Runtime view}
You can use sequence diagrams to describe the way components interact to accomplish specific tasks typically related to your use cases F. 
\subsection{Component interfaces} 
\subsection{Selected architectural styles and patterns}
Please explain which styles/ pattern you used, why, and how
\subsection{Other design decisions }