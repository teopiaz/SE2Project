\section{ALGORITHM DESIGN}
In this section we present some algorithm that we consider relevant for the work of the developers.

\subsection{Find cars near the user}
\label{algoritmo_posizione}
Both Users and Cars's positions are identified in the system by GPS coordinates (Latitude and Longitude).


\noindent When a User $u$ ask for a list of cars the application send to the server his GPS position. \begin{equation}
\vec{u_p} =\begin{bmatrix}user\_latitude \\ user\_longitude\end{bmatrix}
\end{equation}
\\
\noindent The shortest distance (the geodesic) between two given points $P1=(lat1, lon1)$ and $P2=(lat2, lon2)$ on the surface of a sphere with radius R is the great circle distance. 
\\\\
It can be calculated using the following formula:\\
\begin{equation}
D = \arccos\Big(
\sin(lat1) \cdot \sin(lat2) + \cos(lat1) \cdot \cos(lat2) \cdot \cos(lon1 - lon2)
\Big)\cdot R
\end{equation}
\\
\noindent The first step is to convert latitude and longitude in radians:\\
\begin{equation}
lat_{radians} = lat_{degree} \cdot \frac{\pi}{180}
\end{equation}
\\
In order to utilize indices on Lat and Lon we can use a method like the bounding rectangle method in Cartesian space.\\
\noindent We need to define $r$ as the angular radius of a circle in which our cars coordinate will be using $R$ as the Earth radius (6371 km) and $d$ as the maximum distance selected by the user.
\begin{equation}
	 r = \frac{d}{R}  
\end{equation}
for example $$ \frac{5 km}{6371 km} = 0.000784806$$
\\
\noindent Now we need to compute the Minimum and the Maximum Latitude and Longitude:
\begin{eqnarray}
&lat_{min} = lat - r  \\
&lat_{max} = lat + r  \\
&\Delta lon = \arcsin(\frac{\sin(r)}{\cos(lat)})\\
&lat_T = \arcsin(  \frac{\sin(lat)} {\cos(r) }  )\\
&lon_{min} = lon_{T1} = lon - \Delta lon  \\
&lon_{max} = lon_{T2} = lon + \Delta lon 
\end{eqnarray}
\\
\noindent After that we have computed the bounding coordinates we can use them to query a DB.\\
\begin{lstlisting}[language=SQL]
SELECT * FROM Places 
WHERE 
(Lat => $lat_min AND Lat <= $lat_max) AND (Lon >= $lon_min AND Lon <= $lon_max )
HAVING
acos(sin($user_lat) * sin(Lat) + cos($user_lat) * cos(Lat) * cos(Lon - ($user_lon))) <= r;
\end{lstlisting}



\noindent The Java-like code for the managing of the search by the Request Handler:
\begin{lstlisting}[language=Java]
public List search(String address, Position pos){
	List result = new List();
	
	if(address != null){
	Position position = extractCoordinates(address);
	result = reservationManager.getAvailableCars(position); 
	}
	
	else{
	result = reservationManager.getAvailableCars(pos);
	}
	
	return result;
}
\end{lstlisting}

\newpage

\subsection{Reservation}
This is a Java-like code for the algorithm implemented in the Reservation Manager for the addition of a reservation and the management of the 1 hour timer of the reservation:

\begin{lstlisting}[language=Java]
public List reserve(Car c, User u,  String address, Position pos){

 availableCars.delete(c);
 
 reservations.add(new Reservation(u,c);
 
 LocalTime begin = LocalTime.now()
 int beginHour = begin.getHour();
 int BeginMin = begin.getMinute();
 boolean ko = false;
 
 while(!ko){
  	int hourDiff = LocalTime.now().getHour() - beginHour;
  
 	if(hourDiff == 1){
     		int minuteDiff= LocalTime.now().getMinute() - beginMin;
     		if(minuteDiff > 0) {
      		ko = true;		   
     		}
  	}
 }

notifyOberservers(``time_expired'');
}
 


annullaResSeE'FinitoIlTempo();

}
\end{lstlisting}

\newpage

\subsection{Payment, Discounts and Penalties}

This is a Java-like code for the algorithm implemented in the Fee Manager for the payment and the application of discounts and penalties:

\begin{lstlisting}[language=Java]
public void getMoney(Travel t, int distanceFromStation){
	
	//the parameter distanceFromStation is the distance in km from the parking position to the nearest charging station, calculated by the Trip Manager
	
	int discount = 0;
	int penalty = 0;
	int fee = t.getFee();
	
	if(t.passengers >= 2){
	discount = discount + 10;
	}
	
	if(t.car.batteryLevel > 50){
	discount = discount + 20;
	}
	
	if(t.car.isPlugged){
	discount = discount + 20;
	}
	
	if(distanceFromStation > 3){
	penalty = penalty + 30;
	}
	
	fee = fee - ((fee*discount)/100) + ((fee*penalty)/100);
	
	pay(t.user.getpaymentMethod(), fee); //method that will send a request for the payment to the payment node, according to payment method of the user 
	
	
}


\end{lstlisting}