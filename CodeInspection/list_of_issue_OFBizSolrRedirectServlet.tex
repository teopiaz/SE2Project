\todo[inline]{List of issue and issue latex command}


\subsection{Naming Conventions}
\begin{itemize}
\item[1] No issue
\item[2] No issue - No one-character variable is used
\item[3] No issue
\item[4] No issue - There isn't any interface
\item[5] Issue at line 46 - The first letter of the method "DoGet(HttpServletRequest request, HttpServletResponse response)" is capitalized
\item[6] No issue
\item[7] No issue - There isn't any constant
\end{itemize}
\subsection{Indention}
\begin{itemize}
\item[8] No issue - Always three spaces are used for indention
\item[9] No issue - No tabs found in the code
\end{itemize}
\subsection{Braces}
\begin{itemize}
\item[10] No issue - Kernighan and Ritchie style is used consistently.
\item[11] No issue - There are only \textit{if} statements and they are all surrounded by curly braces.
\end{itemize}

\subsection{File Organization}
\begin{itemize}
\item[12] No issue - All sections are separated by a blank line.
\item[13] \begin{itemize}
\item Issue at line 35 - Length = 82 - It is impractical to split the last word.
\item Issue at line 40 - Length = 82 - It is impratical to split the method "getName()"
\item Issue at line 43 - Length = 128 - "javax.servlet.http.HttpServletResponse)" could be put in the following line
\item Issue at line 46 - Length = 119 - impractical to put the \textit{throw} declaration in the following line
\item Issue at line 55 - Length = 119 - impractical to put the \textit{throw} declaration in the following line
\item Issue at line 57 - Length = 83 - impractical to split the method.
\item Issue at line 70 - Length = 87 - impractical to split the method
\item Issue at line 74 - Length = 87 - impractical to split the method
\end{itemize}
\item[14] Issue at line 43 - Length = 128 - See [13]
\end{itemize}
\subsection{Wrapping Lines}
\begin{itemize}
\item[15] No issue - No line breaks
\item[16] No issue - No line breaks
\item[17] No issue - Expressions are all aligned.
\end{itemize}
\subsection{Comments}
\begin{itemize}
\item[18] Comments are used only for briefly explain what the class does and to give a reference to the method that the "doGet" method overrides. There are no comments for the "forwardUrl" method and for blocks of code.
\item[19] No issue - There is no commented out code 
\end{itemize}
\subsection{Java Source Files}
\begin{itemize}
\item[20] No issue
\item[21] No issue
\item[22] No issue
\item[23] No issue
\end{itemize}
\subsection{Package and Import Statements}
\begin{itemize}[24]
	\item \textbf{Lines 31:}\code{ import org.apache.ofbiz.security.Security;	}\\
	\textbf{Unused import.}\\
	\textbf{Solution: } Remove the unused import line.
	
\end{itemize}
\subsection{Class and Interface Declarations}
\begin{itemize}
\item[25]
\begin{itemize}
\item[a] No issue
\item[b] No issue
\item[c] No issue - No implementation comments
\item[d] No issue
\item[e] No issue - No instance variables
\item[f] No issue - No constructors
\item[g] No issue
\end{itemize}
\end{itemize}
\subsection{Initialization and Declarations}

\subsection{Method Calls}
\subsection{Arrays}
\begin{itemize}
	\item No arrays are used in the class behavior.
\end{itemize}
\subsection{Object Comparison}
\todo[inline]{no problem}
\subsection{Output Format}
\todo[inline]{no problem}
\subsection{Computation, Comparisons and Assignments}
\subsection{Exceptions}
\subsection{Flow of Control}
No \code{switch} statement are used in the class.\\
No loop statements are used in the class.
\subsection{Files}
No files are used in the class behavior.