Both the classes assigned to our group belong to the package org.ofbiz.solr.webapp. \textbf{Solr} is a standalone enterprise server with a REST-like API. Solr allows to put documents in the server and to query them. The documents can be put on the server via JSON, XML, CSV or over HTTP. Queries can be done using HTTP GET request method and the result are given in the format used for the document upload. From the name of the package we can easily deduce that the classes are components of the Solr web application.
\par

The functional role of the \textbf{OFBizSolrRedirectServlet} is clearly explained by the javadoc:
\begin{lstlisting}[language=java]
/**
 * OFBizSolrRedirectServlet.java - Master servlet for the ofbiz-solr application.
 */
\end{lstlisting}

The javadoc of the \textbf{TraverseSubContentCacheTransform} says:
\begin{lstlisting}[language=java]
/**
 * TraverseSubContentCacheTransform - Freemarker Transform for URLs (links)
 */
\end{lstlisting}
Apache FreeMarker is a template engine, that is a library that generates text output based on templates. So we can deduce that the TraverseSubContentCacheTransform class is a class for the generation of text representing URLs.	
