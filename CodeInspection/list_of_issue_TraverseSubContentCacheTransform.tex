\todo[inline]{List of issue and issue latex command}


\subsection{Naming Conventions}
\subsection{Indention}
\subsection{Braces}
\subsection{File Organization}
\subsection{Wrapping Lines}
\subsection{Comments}
\subsection{Java Source Files}
\subsection{Package and Import Statements}
\subsection{Class and Interface Declarations}
\subsection{Initialization and Declarations}
\subsection{Method Calls}
\subsection{Arrays}
\subsection{Object Comparison}
\subsection{Output Format}
\subsection{Computation, Comparisons and Assignments}
\subsection{Exceptions}
\begin{itemize}
\item \textbf{Lines 98-105:}
Catch \code{GeneralException} (org.ofbiz.base.util.GeneralException) and retrow a \code{RuntimeException}.
Using such generic exceptions as Error, RuntimeException, Throwable, and Exception prevents calling methods from handling true, system-generated exceptions differently than application-generated errors.\\
\textbf{Solution:} Define an application-specific exception.

\item \textbf{Line 235:} \code{boolean bEquals = contentIdStart.equals(contentIdEnd);}
\textit{NullPointerException might be thrown as 'contentIdStart' is nullable here.}
Solution: Check the \code{null} value before de-reference \code{contentIdStart}.
\end{itemize}
\subsection{Flow of Control}
\begin{itemize}
	\item No \code{switch} statement are used in the class.\\
	No loop statements are used in the class. 
\end{itemize}
\subsection{Files}
\begin{itemize}
	\item No files are used in the class behavior.
\end{itemize}
