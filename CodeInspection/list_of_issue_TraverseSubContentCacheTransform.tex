
\subsection{Naming Conventions}
\subsection{Indention}
\subsection{Braces}
\todo[inline]{no problem}
\subsection{File Organization}
\subsection{Wrapping Lines}
\subsection{Comments}
\subsection{Java Source Files}
\subsection{Package and Import Statements}
\subsection{Class and Interface Declarations}
\subsection{Initialization and Declarations}
\begin{itemize}
	\item \textbf{Lines 52-53:}\code{ public static final String [] upSaveKeyNames [...]
		public static final String [] saveKeyNames = [...]
	 }\\
\textit{ The keyword }\code{final} \textit{for arrays does not reflect to the content of the array. There is no good reason to have a mutable object as} \code{public}. \\
\textbf{Solution: }Make this member \code{protected}. 

\end{itemize}

\subsection{Method Calls}
\todo[inline]{Method Calls}

\subsection{Arrays}
\begin{itemize}
	\item No arrays are used in the class behavior.
\end{itemize}

\subsection{Object Comparison}
\todo[inline]{no problem}

\subsection{Output Format}
\todo[inline]{l'output delle exception?}

\subsection{Computation, Comparisons and Assignments}
\begin{itemize}
	\item \textbf{Line 153:}\code{ if (globalNodeTrail.size() > 0) \{ }\\
\code{isEmpty()}\textit{ generally is O(1),} \code{size()} \textit{is O(n).}\\
	\textbf{Solution:}	Use isEmpty() makes the code more readable and can be more performant. \\
	 
\end{itemize}

\subsection{Exceptions}
\begin{itemize}
\item \textbf{Lines 98-105:}
Catch \code{GeneralException} (org.ofbiz.base.util.GeneralException) and retrow a \code{RuntimeException}.\\
\textit{Using such generic exceptions as Error, RuntimeException, Throwable, and Exception prevents calling methods from handling true, system-generated exceptions differently than application-generated errors.}\\
\textbf{Solution:} Define an application-specific exception.

\item \textbf{Line 235:} \code{boolean bEquals = contentIdStart.equals(contentIdEnd);}
\textit{NullPointerException might be thrown as 'contentIdStart' is nullable here.}\\
\textbf{Solution: }Check the \code{null} value before de-reference \code{contentIdStart}.
\end{itemize}

\subsection{Flow of Control}
\begin{itemize}
	\item No \code{switch} statement are used in the class.\\
	No loop statements are used in the class. 
\end{itemize}

\subsection{Files}
\begin{itemize}
	\item No files are used in the class behavior.
\end{itemize}
