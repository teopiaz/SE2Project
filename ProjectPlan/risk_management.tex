\section{Risk management}
In this section we mention the risks that could affect our project. We provide a brief description of the risk, its probability to actualize, the impact of such actualization on the project and possible solutions to take into account in order to avoid or limit the damages.
The possible values for the probability are \textit{Low, Medium} or \textit{High}, while the possible values for the impact are \textit{Marginal, Serious} or \textit{Catastrophic}.
\subsection{Project Risks}
Project Risks are those risks which threaten the project plan and whose actualization can result in a slipping of the schedule \newline
\begin{adjustbox}{max width=\textwidth}
\begin{tabular}{|l|p{5 cm}|l|l|p{5 cm}|}
\hline
Risk & Description & Probability & Impact & Possible Solutions
\\ \hline
Personnel Shortfall & Due to a wrong estimation of the size of the project, there could be not enough people to complete the project deliveries in time & Medium & Serious & Prepare the personnel to the possibility of extra-work due to the lack of people. \newline Plan a possible new recruitment.	 
\\ \hline
Skill Lack & It could be the case that the team hasn't the skill required to face some types of issues, because it was not supposed to deal with them. & Low & Serious & Recruit people with more skills than the strictly required ones. \newline Consider the organization of update courses. 
\\ \hline
Size of the Project & The project size could be underestimated, because of the lack of knowledge about the field of the project of the staff in charge of planning  & Medium & Marginal & Prepare more resources than the strictly needed ones. \newline Consider a possible extension of the time of the project
\\ \hline
Competitors & Other companies can begin similar projects & Low & Serious & Plan an additional period in the schedule devoted to the development of new features and the improvement of the quality, in order to discourage the development of other similar projects
\\ \hline
Personnel Illness & People of the staff can be ill in critical phases of the project & High & Marginal & Make some tasks of the staff overlap with each other. In this way there will be always possible substitutes to ill people.
\\ \hline
 Data Loss & Data loss during the development & Low & Critical & Backup strategy and software versioning.
\\ \hline
\end{tabular}
\end{adjustbox}
\subsection{Technical Risks}
Technical Risks are those risks which threaten the quality and the functioning of the product and whose actualization makes the implementation more difficult. \newline
\begin{adjustbox}{max width=\textwidth}
\begin{tabular}{|l|p{5 cm}|l|l|p{5cm}|}
\hline
Risk & Description & Probability & Impact & Possible Solution
\\ \hline
Obsolete Design & Some new technology, that provides so as many better features to force the team to change the design of the system, could born in the course of the project & Low & Serious & Design the system in a way that allows to change components easily.
\\ \hline
Inadequate Infrastructure & The resources, the facilities and the machines provided by the company or owned by the team are discovered to be inadequate for the project & Low & Catastrophic & While defining the budget with the customer, ask him the creation of a fund for possible substitutions of inadequate equipment.
\\ \hline
New Requirements & New requirements are defined after the design phase& Low & Serious &The Waterfall development approach require frozen requirements. Change the to Scrum or Agile approach.

\\ \hline
\end{tabular}
\end{adjustbox}
\subsection{Business Risks}
Business Risks are those risks which threaten the feasibility of a complete product and the chances of selling it. Their actualization can cause bad results on sales and evaluation of consumers. \newline
\begin{adjustbox}{max width=\textwidth}
\begin{tabular}{|l|p{5 cm}|l|l|p{5 cm}|}
\hline
Risk & Description & Probability & Impact & Possible Solution
\\ \hline
Budget Lack & At some point it can be discovered that there has been an underestimation of the costs of project and that the budget is insufficient & Medium & Serious & Prepare a document to deliver to the customer, in which are explained the good results obtained till that moment, the reason for the budget increase, the disadvantages of non-concluding the project and the advantages of improving it with more functionalities.
\\ \hline
Market Risk & The team has developed a good product, but this does not match the real requirements, expectations and needs of consumers, resulting in poor sales & Medium & Catastrophic & Consider interviews to stakeholder and random consumers during all the phases of project in order to have feedbacks on the partial work done till a certain moment.
\\ \hline
\end{tabular}
\end{adjustbox}