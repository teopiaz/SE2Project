\section{INTRODUCTION}
\subsection{Purpose and Scope}
\subsubsection{Purpose}
The purpose of this document is to provide an estimation of the cost and the size of the project. This knowledge is needed to organize the resources needed in the development of the system and to properly schedule the activities of the project.\newline
To reach this purpose, two models will be used:
\begin{itemize}
\item \textbf{Function Point:} This approach is used to estimate the size of the project in LOC.
\item \textbf{COCOMO:} This model is used to estimate the effort required by the project in PH
\end{itemize}

\subsubsection{Scope}
PowerEnJoy is a digital system for the management of a car-sharing service that only employs electric cars. In particular the aim is to develop a mobile application that allows the user to pick up and use electric cars in the areas reached by the service.

\subsection{List of Definitions and Abbreviations} 
\subsubsection{Acronyms}
\begin{itemize}
\item RASD: Requirement Analysis and Specification Document.
\item DD: Design Document.
\item DBMS: Database Management System.
\item DB: Database.

\item GPS: Global Positioning System

\end{itemize}
\subsubsection{Definitions}

\subsubsection{Abbreviations}
\begin{itemize}
	\item  FP: Function Points.
	\item  ILF: Internal logic file
	\item  ELF: External logic file.
	\item  EI: External Input.
	\item  EO: External Output.
	\item  EQ: External Inquiries.
	\item PH: Person Hours.
	\item LOC: Lines Of Code.
\end{itemize}



\subsection{List of Reference Documents}
\begin{itemize}
	\item PowerEnJoy RASD.	
	\item PowerEnJoy DD.
	\textbf{NOTE:} The reader must refer to the version 1.1 of the DD.
\end{itemize}

