\section{INTEGRATION STRATEGY}
\subsection{Entry Criteria} 
Before the integration testing can begin, both the RASD and the DD must be completed. Furthermore at least the 80\% of the lines of the code of all components must have been tested with a Unit Test. We strongly suggest the Junit testing framework for Unit Testing.
\subsection{Elements to be Integrated} Identify the components to be integrated, refer to the design document to identify such components in a way that is consistent with your design
\newline 

As we have described in the DD, the main components of the system are:
\begin{itemize}
\item Web Client GUI
\item Mobile Client GUI
\item Driver Client GUI
\item Service Manager 
\item HTTP Server
\item Request Handler
\item REST Parser
\item DBMS
\end{itemize}

In turn, the Service Manager is composed of:
\begin{itemize}
\item Car Manager
\item Trip Manager
\item User Manager
\item Position Manager
\item Fee Manager
\end{itemize}
\noindent
For a further description of the components refer to the Design Document. 
\newline 
Of course, because before the Service Manager can be integrated with other components, the components that compose it have to be integrated.


\subsection{Integration Testing Strategy} Describe the integration testing approach (top-down, bottom-up, functional groupings, etc.) and the rationale for the choice of that approach.
\subsection{Sequence of Component/Function Integration} 
\subsubsection{Software Integration Sequence} For each subsystem, identify the sequence in which the software components will be integrated within the subsystem; relate this sequence to any product features that are being built up
\subsubsection{Subsystem Integration Sequence} Identify the order in which subsystems will be integrated;  if you have a single subsystem, 2.4.1 and 2.4.2 have to be merged in a single section





